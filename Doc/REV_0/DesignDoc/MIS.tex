\documentclass[12pt, titlepage]{article}
\usepackage[letterpaper, portrait, margin=1in]{geometry}

\usepackage{amsmath}
\usepackage[T1]{fontenc}
\usepackage{babel}
\usepackage{tabularx}
\usepackage{textcomp}
\usepackage{titlesec}
\setcounter{secnumdepth}{4}
\usepackage{hyperref}
\usepackage{xcolor}
\usepackage{booktabs}
\usepackage{placeins}
\usepackage{multirow}


\hypersetup{
    bookmarks=true,         % show bookmarks bar?
      colorlinks=true,       % false: boxed links; true: colored links
    linkcolor=black,          % color of internal links (change box color with linkbordercolor)
    citecolor=green,        % color of links to bibliography
    filecolor=magenta,      % color of file links
    urlcolor=cyan           % color of external links
}

\title{SE 3XA3: Module Internal Specification\\SnakeGame Project}

\author{Team L03G09
		\\ Qiang Gao  (gaoq20)
		\\ Zhiwei Li  (liz342)
		\\ Longwei Ye (yel16)
}

\date{\today}


\begin{document}



\maketitle

\newpage

\tableofcontents
\newpage

\section{Module Hierarchy}
\begin{table}[h!]
\centering
\begin{tabular}{p{0.3\textwidth} p{0.6\textwidth}}
\toprule
\textbf{Level 1} & \textbf{Level 2}\\
\midrule

{Hardware-Hiding Module} & ~ \\
\midrule

\multirow{7}{0.3\textwidth}{Behaviour-Hiding Module} & View Module\\
& BGM Control Module\\
& Inputs Checking Module\\
& Game Control Module\\
\midrule

\multirow{3}{0.3\textwidth}{Software Decision Module} & SnakeGame Module\\
& History Module\\
& Snake Color Module\\
\bottomrule

\end{tabular}
\caption{\textbf{Module Hierarchy}}
\label{TblMH}
\end{table}
\section{MIS of BGM Control Module}
		\subsection{Interface Syntax}
		\subsubsection{Exported Access Programs}
		\begin{tabular}[pos]{|c|c|c|c|}
			
			\hline
			%	\label
			\textbf{Name}& \textbf{In} & \textbf{Out} & \textbf{Exceptions} \\ \hline
			BgmControl & - & - & -\\ \hline
			SoundChange & - & - & -\\ \hline

			
		\end{tabular}
		
		\subsection{Interface Semantics}
		\subsubsection{State Variables}
	    music: object \\
		currentButton: object \\
		sound: object \\
		span: object \\
		music: object
		\subsubsection{Environmental Variables}
		N/A
		\subsubsection{Assumptions}
		Variables should be set before trying to access them
		
		\subsubsection{Access Program Semantics}
		

		BgmControl():
		
		Input: None
		
		Transition: The BGM will be paused or played by clicking the button.
		
		Output: None
		
		Exception: None\\ 
		\\
		SoundChange():
		
		Input: None
		
		Transition: The volume of BGM will be turned up or down by sliding the slider.
		
		Output: None
		
		Exception: None\\ 
		\\
		
\section{MIS of Inputs Checking Module}
	\subsection{Interface Syntax}
		\subsubsection{Exported Access Programs}
		
	\begin{tabular}[pos]{|c|c|c|c|}
	\hline
	\textbf{Name}& \textbf{In} & \textbf{Out} & \textbf{Exceptions} \\ 
	\hline
	checkColorHex & - & Boolean & -\\ 
	\hline
	checkUserName & - & Boolean & -\\ 
	\hline
	
					
	\end{tabular}		
		
	\subsection{Interface Semantics}
		\subsubsection{State Variables}
		color\_HEX: object
		txtUserName: object
		\subsubsection{Environmental Variables}
		N/A
		\subsubsection{Assumptions}
Variables should be set before trying to access them

		\subsubsection{Access Program Semantics}
		
		checkColorHex():
		
		Input: None\\
		
		Output: True if the length is valid, false if it is not.\\
		
		Exception: None\\
		\\
		checkUserName():
		
		Input: None\\
		
		Output: True if the length is valid, false if it is not.\\
		
		Exception: None\\
		\\

\section{MIS of Game Control Module}
	\subsection{Interface Syntax}
		\subsubsection{Exported Access Programs}
		\begin{table}[!htbp]
				\begin{tabular}{|c|c|c|c|}
						\hline
	\textbf{Name}& \textbf{In} & \textbf{Out} & \textbf{Exceptions} \\ 
	\hline
					onCountChange & integer, String & - & Invalid Input \\ \hline
					onGamePause & boolean & .mp4 & - \\ \hline
					onGameOver & integer & - & - \\ \hline
					btnStart.onclick & event & - & - \\ \hline
					btnPasue.onclick & event & - & - \\ \hline
					updateSnake & - & updateSnake & -\\ \hline
				\end{tabular}
			\end{table}
	\subsection{Interface Semantics}
		\subsubsection{State Variables}
		btnStart: object
		btnPasue: object
		snaColor: object
		gameHistory: String[]
		gameSnake: SnakeGame
		\subsubsection{Environmental Variables}
		
		N/A
		
		\subsubsection{Assumptions}
		
		All variables are set when the object is constructed.

		\subsubsection{Access Program Semantics}

	  onCountChange():
		
		Input: integer, Scores users got\\
		
		Transition: Scores changing when user playing the game\\
		
		Exception: None\\
		\\
	  onGamePause():
		
		Input: boolean, status\\
		
		Transition: play and pause the game\\
		
		Exception: None\\
		\\
	onGameOver():
		
		Input: integer, scores\\
		
		Transition: Display "game over" and scores\\
		
		Exception: None\\
		\\
		btnStart.onclick(event):
		
		Input: click\\
		
		Transition: start the game\\
		
		Exception: None\\
		\\
		btnPasue.onclick(event):
		
		Input: click\\
		
		Transition: pause the game\\
		
		Exception: None\\
		\\
		updateSnake():
		
		Input: None\\
		
		Transition: Update the status of game\\
		
		Output: updateSnake\\
		
		Exception: None\\
		\\

\section{MIS of SnakeGame Module}
	\subsection{Interface Syntax}
		\subsubsection{Exported Access Programs}
	
	\begin{tabular}[pos]{|c|c|c|c|}
	\hline
	\textbf{Name}& \textbf{In} & \textbf{Out} & \textbf{Exceptions} \\ 
	\hline
	SnakeGame & object, object & SnakeGame & -\\ 
	\hline
	initSnake & - & - & -\\ 
	\hline
	initScense & - & - & - \\ 
	\hline
	genFood & - & - & -\\ 
	\hline
	genSpeeder & - & - & -\\ 
	\hline
	eatFood & object & boolean & -\\ 
	\hline
	eatSpeeder & object & boolean & -\\ 
	\hline
	gameOver & - & boolean & -\\ 
	\hline
	snakeMove & - & - & -\\ 
	\hline
	changeSpeed & - & - & -\\ 
	\hline
	handleKeyInput & object & - & -\\ 
	\hline
	initGame & - & - & -\\ 
	\hline
	triggerEvent & object,object & - & -\\ 
	\hline
	runGame & - & - & -\\ 
	\hline
	pauseGame & - & - & -\\ 
	\hline
	changeGameStatus & - & - & -\\ 
	\hline
	startGame & - & - & -\\ 
	\hline
	
					
	\end{tabular}			
		
	\subsection{Interface Semantics}
		\subsubsection{State Variables}
		gameScense: object\\
		graphic: object\\
		count: integer\\
		itemCount: integer\\
		itemvalid: boolean\\
		snake: object\\
		curFood: object\\
		curSpeeder: object\\
		runId: Integer\\
		isMoved: boolean\\
		gameStatus: boolean\\
		curDirection: integer\\
		size: integer\\
		rowCount: integer\\
		colCount: integer\\
		snakeColor: object\\
		foodColor: object\\
		speederColor: object\\
		scenseColor: object\\
		directionKey: double\\
		pauseKey: double\\
		levelCount: integer\\
		curSpeed: double\\
		onCountChange: object\\
		onGamePause: object\\
		onGameOver: object
		\subsubsection{Environmental Variables}
		N/A
		\subsubsection{Assumptions}
		All variables are set when the object is constructed.
		\subsubsection{Access Program Semantics}
		SnakeGame(gameScenseId, gameConfigObj):
		
		Input: The id for the current game and combined variables used for a game
		
		Transition: None
		
		Output: Construct a SnakeGame
		
		Exception: None \\
		\\
		initSnake():
		
		Input: None
		
		Transition: store the initial position of a snake in game
		
		Exception: None\\
		\\
		initScense():
		
		Input: None
		
		Transition: print the initial background of the game
				
		Exception: None\\
		\\
		genFood():
		
		Input: None
		
		Transition: print a common food in a random position
				
		Exception: None\\
		\\
		genSpeeder():
		
		Input: None
		
		Transition: print a speeder in a random position
				
		Exception: None\\
		\\
		eatFood(snakeHead):
		
		Input: snakeHead, an object
		
		Output: True if a common food is eaten by the snake, false otherwise
				
		Exception: None\\
		\\
        eatSpeeder(snakeHead):
		
		Input: snakeHead, an object
		
		Output: True if a speeder is eaten by the snake, false otherwise
				
		Exception: None\\
		\\
		gameOver():
		
		Input: None
		
		Output: true if the game is ended, false otherwise
				
		Exception: None\\
		\\
		snakeMove():
		
		Input: None
		
		Transition: Take keyboard input to update one position shift, check 
if any food or speeder is eaten, and if speed is needed to change.
				
		Exception: None\\
		\\
		changeSpeed():
		
		Input: None
		
		Transition: refresh speed based on current score. 
				
		Exception: None\\
		\\
		handleKeyInput(key):
		
		Input: key on keyboard, an object
		
		Transition: Change moving direction 
based on key.
				
		Exception: None\\
		\\
		initGame():
		
		Input: None
		
		Transition: initialize the game
				
		Exception: None\\
		\\
		triggerEvent(callback,argument):
		
		Input: callback,argument
		
		Transition: template the end game, pause game events.
				
		Exception: None\\
		\\
		runGame():
		
		Input: None
		
		Transition: continue the game or end the game.
				
		Exception: None\\
		\\
		pauseGame():
		
		Input: None
		
		Transition: pause the game based on the trigger runId.
				
		Exception: None\\
		\\
		changeGameStatus():
		
		Input: None
		
		Transition: change the game status pause or continue, based on the trigger runId
				
		Exception: None\\
		\\
		startGame():
		
		Input: None
		
		Transition: do the initial part of the game, and set gameStatus
				
		Exception: None\\
		\\

\section{MIS of History Module}
	\subsection{Interface Syntax}
		\subsubsection{Exported Access Programs}
		\begin{table}[!htbp]
		\begin{tabular}{|c|c|c|c|}
		\hline
	\textbf{Name}& \textbf{In} & \textbf{Out} & \textbf{Exceptions} \\ 
	\hline
			displayHistory & - & - & - \\ \hline
			pushHistory & Integer & - & - \\ \hline
		\end{tabular}
	\end{table}
		
	\subsection{Interface Semantics}
		\subsubsection{State Variables}
		playerName: object\\
		playerScore: Integer \\
		toWrite: object \\
		gameHistory: object
		
		\subsubsection{Environmental Variables}
		N/A
		\subsubsection{Assumptions}
		All variables are set when the object is constructed. 

		\subsubsection{Access Program Semantics}

        displayHistory():

		Input: None
		
		Transition: Display the history of the game.
		
		Output: None
		
		Exception - None\\
		\\
		pushHistory(count):
		
			Input: Scores user gets
			
			Transition: Record scores in gameHistory
			
			Output: None
			 
			Exception - None\\
			\\


	
\section{MIS of Snake Colors Module}
		\subsection{Interface Syntax}
		\subsubsection{Exported Access Programs}
		\begin{tabular}[pos]{|c|c|c|c|}
			
			\hline
			%	\label
			\textbf{Name}& \textbf{In} & \textbf{Out} & \textbf{Exceptions} \\ \hline
			snaColorR.onclick & event & - & -\\ \hline
			snaColorG.onclick & event & - & -\\ \hline
			snaColorB.onclick & event & - & -\\ \hline
			snaColorDefault.onclick & event & - & -\\ \hline
			snaColorSet.onclick & event & - & -\\ \hline
		\end{tabular}
		
		\subsection{Interface Semantics}
		\subsubsection{State Variables}
		N/A
		
		\subsubsection{Environmental Variables}
		N/A
		
		\subsubsection{Assumptions}
		None
		
		\subsubsection{Access Program Semantics}
		snaColorR.onclick():
		
		Input: event\\
		
		Transition: Red color\\
		
		Output: None\\
		
		Exception: None\\
		\\
		snaColorG.onclick():
		
		Input: event\\
		
		Transition: Green color\\
		
		Output: None\\
		
		Exception: None\\
		\\
		snaColorB.onclick():
		
		Input: event\\
		
		Transition: Blue color\\
		
		Output: None\\
		
		Exception: None\\
		\\
	    snaColorDefault.onclick():
		
		Input: event\\
		
		Transition: Default white color\\
		
		Output: None\\
		
		Exception: None\\
		\\
		snaColorSet.onclick():
		
		Input: event\\
		
		Transition: Change color pf snake\\
		
		Output: None\\
		
		Exception: None\\
		\\
	
	
\section{Side Notes}
The MIS document consists of two files:
\begin{itemize}
	\item This MIS.pdf file
	\item Automated generated document using open-source generator \emph{JSDoc}, with reference link: \url{https://jsdoc.app}. 
	The generated html file is located at src/REV\_0/js/out folder. Please refer for details.
\end{itemize}

\section{Revision History}
\begin{table}[bp]
\begin{tabularx}{\textwidth}{p{3cm}p{2cm}X}
\toprule {\bf Date} & {\bf Version} & {\bf Notes}\\
\midrule
2022/3/15 & 1.0 & MIS for Snake Color Module\\
2022/3/16 & 1.1 & MIS for BGM Control Module \& Game History Module\\
2022/3/17 & 1.2 & MIS for Inputs Checking Module\\
2022/3/18 & 1.3 & MIS for Game Control Module\\
2022/3/18 & 1.4 & Revise for MIS document\\
\bottomrule
\end{tabularx}
\caption{\bf Revision History}
\end{table}

			
\end{document}
